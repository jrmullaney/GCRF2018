\documentclass[11pt]{article}
\usepackage[a4paper,bindingoffset=0cm,%
            left=2cm,right=2cm,top=2cm,bottom=2cm,%
            footskip=0.5cm]{geometry}
\usepackage[pdftex]{graphicx}
\usepackage{multicol}

%\usepackage{psfig}

\usepackage{siunitx,amsmath,amssymb}

\usepackage{sectsty}
\sectionfont{\fontsize{12}{15}\selectfont}
\subsectionfont{\fontsize{11}{15}\selectfont}

%\renewcommand*\sfdefault{phv}
%\renewcommand{\familydefault}{\sfdefault}
\makeatletter
\newenvironment{tablehere}
  {\def\@captype{table}}
  {}

\newenvironment{figurehere}
  {\def\@captype{figure}}
  {}
\makeatother

\begin{document}
%\title{Broadening the impact of astronomical data handling\vspace{-2ex}}
%\maketitle
\setcounter{figure}{0}
\noindent
{\large \bf From Stars to Baht: Phase II}

\noindent
{\large \bf Response to reviewers' reports}

\vspace{2mm}
\noindent
We thank all the reviewers for their constructive feedback on our project, and the panel for giving us the opportunity to respond to their feedback. In what follows, we refer to reviewers 022396196, 057124662 and 039760429 as 02, 03 and 05, respectively.

\vspace{2mm}
\noindent
It is clear that the greatest concern of reviewers 02 and 05 is the apparent lack of support on the UK-side. To address this specific concern, since the submission of the funding application we have garnered offers of support from a significant numbers of UK-based academics and support staff, all of whom possess skills highly relevant to the project. Each have agreed to act as consultants on the project, offering advice where necessary throughout the two-year period. Our consultants include members of the University of Sheffield based in the following Departments/Schools/Teams:
\begin{table}[h]
\begin{tabular}{lll}
  Name & Department/School/Team & Relevant expertise\\
  \hline
  Dr. Mauricio Alvarez Lopez & Computer Science & Machine learning \\
  Prof. Eleni Vasilaki & Computer Science & Machine learning\\
  Mr. David Holloway & Digital Learning Team & MOOC production \\
  Dr. Erica Ballantyne & Management & Operations and supply chain \\
  Dr. Miguel Juarez & Maths \& Statistics & Bayesian methods in Big Data analysis \\
  Dr. Kevin Walters &  Maths \& Statistics & Bayesian methods in Big Data analysis \\
  Prof. Richard Wilkinson & Maths \& Statistics & Statistical analysis of complex systems\\
  Prof. Simon Goodwin & Physics and Astronomy & Big Data analytics \\
  Dr. Richard Parker & Physics and Astronomy & Big Data analytics \\
\hline
\end{tabular}  
\end{table}  

\noindent
The members of Computer Science in the above table are also part of Sheffield's Machine Learning Research Network, whose role is to support researchers and students in using and developing Machine Learning. Further, Prof. Simon Goodwin and Dr. Richard Parker have recently been collaborating with Dr. Erica Ballantyne on research projects focussed on the optimisation of transport networks; experience that is particularly relevant to the Air Asia and Thanapirya sub-projects. Mr. David Holloway is a member of Sheffield's in-house, centrally-funded MOOC-production team, and has agreed to offer his advice on developing our MOOC. We do not require any additional funds for our consultants.

\vspace{2mm}
\noindent
The role of our consultants will be to offer advice and guidance to the students and their supervisors during the course of the project. As noted in our Gantt Chart, the core team members will write a detailed progress report every 6 months. This will be disseminated among all our consultants, giving them the opportunity to offer suggestions to the students (e.g., "this sounds like an X problem, you may want to try Y", "you may wish to read this paper..."). As such, the project will remain exactly as stipulated in our application, with support and advice from more leading UK experts. 

\vspace{2mm}
\noindent
In response to the question of the scalability of the project raised by reviewer 03, we'd like to inform the panel that, since the submission of the application, our team has held a networking event at MFU attended by representatives from {\it all} the proposed external partners highlighted in our application. Most of these representatives travelled from Bangkok ($\sim$1.5hr flight), or even further afield, thereby demonstrating their enthusiasm to be involved in the project. In addition, the goal of developing a MOOC is to enable the project to have a truly national impact. After developing the MOOC in consultation with our proposed external partners and UK experts, we will advertise the free resource to employees of other businesses and organisations within Thailand. On completion of the MOOC, these employees can attend one of our residential workshops to gain further training. This will also raise awareness of the Masters programme at MFU, which will continue to work with additional external partners on new projects beyond the end of the GCRF funding.

\vspace{2mm}
\noindent
In answer to reviewer 05's question about who and how the MOOC will be developed. This will be done by the core team members with outsourced design \& filming, and in consultation with Sheffield's MOOC production unit (see table). To date, this unit has produced 16 MOOCs, and have agreed to share with our team their knowledge and experience of how to develop a successful MOOC.  

\vspace{2mm}
\noindent
Finally, reviewer 03 raises the prospect of a case study. We expect to convert our work with our current partner, Thanapirya, into a case study. Our team of five students have delivered to Thanapirya's Management Team their initial model that predicts their best-selling products given the demographics of the local area. Thanapirya have provided feedback and the students are now working toward a mid-December delivery of a live, online system that can be used to help choose future trading locations. Once the students have received their final feedback, we will include Thanapirya as a case study in our final report for the 2017 GCRF project (due March, 2019). 

\end{document}

